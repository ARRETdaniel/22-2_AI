% ----------------------------------------------------------

\chapter[Fundamentação Teórica]{Fundamentação Teórica}
\begin{enumerate}
   \item  Como contribuição teórica inicial para o nosso projeto, temos o material do canal no Youtube 'Python Engineer' a playlist de vídeo aulas,  'Chat Bot With PyTorch - NLP Beginner Tutorial' \footnote{\url{https://www.youtube.com/playlist?list=PLqnslRFeH2UrFW4AUgn-eY37qOAWQpJyg}}. A partir desse conteúdo poderemos dar início aos nossos desenvolvimentos e pesquisas. No tutorial, é desenvolvido um chatbot simples usando PyTorch e Deep Learning. Também fornecendo uma introdução a algumas técnicas básicas de Processamento de Linguagem Natural (PLN).

   \item Seguiremos, também, os conteúdos já aprendidos nas aulas de  'Inteligência Artifical 2022/2 - UENF' para nos auxiliar na formulação teórica do nosso projeto.

   \item  Salientamos que com o decorrer do projeto nossas referências de materiais, para esse projeto, tenderá a aumentar devido a novas descobertas.
\end{enumerate}







\chapter[Metodologia]{Metodologia}

Baseado no “Project-based learning” \cite{krajcik2006project}. Seguiremos os estudos através de um projeto que aborda problemas do mundo real, cujo muitos não tem resposta única. Ao longo desse projeto será possível fazer novas perguntas e encontrar suas possíveis respostas por meio de uma investigação sustentada.


\vspace {1mm}

Este Plano de Pesquisa também utilizará as seguintes metodologias:
\begin{itemize}
   \item \textit{Pesquisa Exploratória; visando promover o enriquecimento do conhecimento sobre os diferentes assuntos relacionados a IA, ML, e Chatbots:
         }
         \begin{itemize}
            \item \textit{Levantamento Bibliográfico;}
            \item \textit{Levantamento documental;}
            \item \textit{Minicursos e Vídeo aulas;}
            \item \textit{Obtenção de experiências.}
         \end{itemize}
\end{itemize}
